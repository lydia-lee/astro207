\qns{Flat Disks}

\work{
	For an annulus in the ring at distance $r$, power in must equal power out
	\begin{align*}
		P_\text{in} &= P_\text{out}\\
		F_\text{in} &= F_\text{out}\\
			&= \sigma T^4
	\end{align*}
	where $T$ is the temperature in question.
	For the flux coming from the star where $\tan\theta_c = \frac{R_*}{r}$,
	\begin{align*}
		F_\text{in} &\approx F_*\cdot 4\pi R_*^2 \cdot \frac{1}{4\pi r^2} \cdot C\sin\theta_c\\
			&\approx \sigma T_*^4\frac{R_*^2}{r^2}\frac{R_*}{r}
	\end{align*}
	where $C$ (roughly 1) accounts for error from $r-R_* \approx r$ and the fact that there should be an integral over $\theta$ going from $0$ to $+\theta_c$.
	\begin{align*}
		\sigma T^4 &\approx \sigma T_*^4\frac{R_*^2}{r^2}\frac{R_*}{r}\\
		T &\approx T_*\left(\frac{R_*}{r}\right)^\frac{3}{4}
	\end{align*}
	\textcolor{red}{This feels weird, but I guess as you get further away a slice of the ring of the same area occupies a smaller solid angle from the view of some $z\neq 0$ point on the star?}}

\ans{
	$$T \approx T_*\left(\frac{R_*}{r}\right)^\frac{3}{4}$$
}