\qns{Blackbody Flux}

\ans{
	\begin{align*}
		B_\nu &= \frac{2h\nu^3}{c^2}\frac{1}{e^{\frac{h\nu}{k_BT}}-1}\\
		\int_{0}^\infty\int_{0}^{2\pi}\int_{0}^{\pi/2}B_\nu\cos\theta\sin\theta d\theta d\phi d\nu &= 2\pi\int_{0}^{\pi/2}\cos\theta\sin\theta d\theta \int_{0}^\infty B_\nu d\nu\\
			&= 2\pi \cdot \frac{1}{2} \cdot\int_0^\infty \frac{2h\nu^3}{c^2}\frac{1}{e^\frac{h\nu}{k_BT} - 1}d\nu\\
			&= \pi \int_0^\infty \frac{2h\frac{c^3}{\lambda^3}}{c^2}\frac{1}{e^\frac{hc}{\lambda k_BT}-1} d\nu \\
			&= \pi \int_0^\infty \frac{2hc^2}{\lambda^5}\frac{1}{e^\frac{hc}{\lambda k_BT}-1} d\lambda \longleftarrow \begin{cases}\nu = \frac{c}{\lambda}\\d\nu = -\frac{c}{\lambda^2}d\lambda\end{cases}\\
			&= \pi \cdot 2hc^2 \cdot \frac{hc}{k_BT} \cdot \left(\frac{k_BT}{hc}\right)^5 \int_0^\infty \frac{1}{x\left(e^{1/x}-1\right)}dx \longleftarrow \begin{cases}x = \lambda \frac{k_BT}{hc}\\dx = d\lambda\frac{k_BT}{hc}\end{cases}\\
			&= \pi \cdot \frac{2k_B^4T^4}{h^3c^2}\cdot \frac{\pi^4}{15}\\
			&= \frac{2\pi^5k_B^4}{15h^3c^2}T^4\\
			&= \sigma T^4
	\end{align*}
\textcolor{white}{And of course I forgot the Jacobian the first time around}}

\sol{
	To derive the flux from the Planck Distribution, note that flux has units of erg.\si{\second^{-1}.\centi\meter^{-2}} while the distribution has units of erg.\si{\second^{-1}.\centi\meter^{-2}.\hertz^{-1}.\steradian^{-1}}. So clearly we need to integrate out the frequency and angular dependence.
	The only catch is that we also need to account for the detector seeing the full blackbody patch directly aove the radiating patch (at $\theta = 0$) but seeing the effective vanishing area ($dA = 0$) off to the side ($\theta = \pi$) ()
	\begin{figure}[t]
		\centering
		\caption{\textcolor{red}{TODO figure}}
		\label{fig:blackbody_flux}
	\end{figure}
	Hence, I need to toss in an extra factor of $2\cos\theta$ to get this right.
	And since one side is cold, I only integrate $\theta$ up to $\theta = \pi$.
	Let $\Phi$ be the frequency integral of the Planck distribution at a certain temperature.
	Then:
	\begin{align*}
		\Phi &= \int_0^\infty B_\nu(T)d\nu\\
			&= \int_0^\infty \frac{2h\nu^3}{c^2}\frac{d\nu}{e^\frac{h\nu}{k_BT} - 1}\\
			&= \frac{2h}{c^2}\left(\frac{k_BT}{h}\right)^4\int_0^\infty \frac{x^5}{e^{1/x} - 1}dx \longleftarrow \begin{cases}
					x &= \frac{k_BT}{h\nu}\\
					dx &= -d\nu \frac{k_BT}{\nu^2}
				\end{cases}\\
			&= \frac{2\pi^4}{15}\frac{k_B^4T^4}{c^2h^3} \left[\frac{\text{erg}}{\si{\second\times\centi\meter^2\times\steradian}}\right]
	\end{align*}
	So the total flux is
	\begin{align*}
		F &= \int B_\nu(T)d\nu\cos\theta d\Omega\\
			&= \Phi \int_0^{2\pi}d\phi\int_0^{\frac{\pi}{2}}cos\theta\sin\theta d\theta\\
			&= \pi \Phi \left[\frac{\text{erg}}{\si{\second\times\centi\meter^2}}\right]
	\end{align*}
	Plugging in for $\Phi$:
	\begin{align*}
		F &= \sigma T^4, \sigma \equiv \frac{2\pi^5}{15}\frac{k_B^4}{c^2h^3}
	\end{align*}
}