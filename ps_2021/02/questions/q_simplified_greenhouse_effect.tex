\qns{A Simplified Greenhouse Effect}

\work{
	\textcolor{red}{I suppose we'll assume the atmosphere doesn't emit as a blackbody and Earth doesn't reflect anything coming off of the sun?}
	\begin{center}
		\begin{tikzpicture}
		% atmosphere
		\draw[dashed,line width=0.25mm] (-1,3) coordinate (atTopLeft) 
			-- ++(4,0) coordinate (atTopRight);
		\draw[dashed,line width=0.25mm] (atTopLeft)++(0,-1) coordinate (atBotLeft) 
			-- (atBotLeft-|atTopRight) coordinate (atBotRight);
		\draw[<->] (atTopLeft) -- (atBotLeft);
		\draw ($(atTopLeft)!0.5!(atBotLeft)$) node[anchor=east] () {$z_\text{atm}$};
		
		% ground
		\draw[line width=0.25mm] (atBotLeft)++(0,-2) coordinate (gndLeft) 
			-- (gndLeft-|atTopRight) coordinate (gndRight) node[anchor=west] () {$T_\text{gnd}$};
		
		% from sun
		\draw[->] ($(atTopLeft)!0.2!(atTopRight)$)++(0,1) coordinate (sunStart) 
			-- (sunStart|-gndLeft) coordinate (sunStop);
		\draw (sunStart) node[anchor=east] () {$F_\text{sun}$}; 

		% earth blackbody
		\draw[->] ($(gndRight)!0.2!(gndLeft)$) coordinate (earthStart)
			-- (earthStart|-atBotRight) coordinate (earthStop);
		\draw (earthStop) node[anchor=south] () {$F_\text{Earth}$};

		% scattering
		\draw[->] ($(atBotLeft)!0.5!(atBotRight)$) coordinate (scatStart)
			-- (scatStart|-gndLeft) coordinate (scatStop);
		\draw (scatStart) node[anchor=south] () {$F_\text{scat}$};

		% top of atmosphere
		\draw[->] (atTopRight-|earthStart) coordinate (outStart)
			-- (outStart|-sunStart) coordinate (outStop);
		\draw (outStop) node[anchor=west] () {$F_\text{out}$};
\end{tikzpicture}
	\end{center}
	\begin{align*}
		F_\text{Earth} = \sigma T_\text{gnd}^4 &= F_\text{scat} + F_\text{sun}\\
		\alpha &= n\sigma_\text{scat} = \frac{N}{z_\text{atm}}\sigma_\text{scat}
	\end{align*}
	We consider the amount of energy scattered back by a volume of scatterers depth $z$.
	\begin{multicols}{2}
		\raggedcolumns
		\begin{center}optically thin\end{center}
		$$z = z_\text{atm}$$
		Here we treat some constant fraction of energy which doesn't reach depth $s \in (0,z_\text{atm}]$ can be as scattered and assume extinction along the back-path isn't significant given the short distance.
		\vfill\columnbreak
		\begin{center}optically thick\end{center}
		$$z = \lambda_\text{mfp} = \frac{1}{n\sigma_\text{scat}} = \frac{z_\text{atm}}{N\sigma_\text{scat}}$$
		This isn't an unreasonable assumption, given backscattering within the atmosphere necessarily deals with extinction with the decaying exponential on the path back out of the atmosphere. 
	\end{multicols}
	\begin{multicols}{2}
		\raggedcolumns
		\begin{align*}
			F_\text{scat} &= F_\text{Earth}\left(1-k_\text{thin}e^{-\alpha z_\text{atm}}\right), 0 < k_\text{thin} < 1\\
			F_\text{Earth} &= F_\text{Earth}\left(1-k_\text{thin}e^{-\alpha z_\text{atm}}\right) + F_\text{sun}\\
			\sigma T_\text{gnd}^4 &\propto e^{N\sigma_\text{scat}}
		\end{align*}
		\vfill\columnbreak
		\begin{align*}
			F_\text{scat} &= F_\text{Earth}(1-k_\text{thick}), 0 < k_\text{thick} < 1/e\\
			F_\text{Earth}(k_\text{thick}) &= F_\text{sun}\\
			T_\text{gnd}^4 &\propto \text{constant}
		\end{align*}
	\end{multicols}
	% Considering $j_{\nu,\text{scat}} = n\sigma_\text{scat}J_\nu = \frac{N}{z_\text{atm}}\sigma_\text{scat}J_\nu$,
	% $$j_{\nu,\text{scat}}V_\text{scat} \propto \begin{cases}NJ_\nu & \text{optically thin}\\ J_\nu & \text{optically thick}\end{cases}$$
	% where $J_\nu = KF_\text{Earth}$ for some scalar $K$.
	% \begin{align*}
	% 	F_\text{Earth} &= F_\text{scat} + F_\text{sun}\\
	% 	\sigma T_\text{gnd}^4 &= F_\text{scat} + F_\text{sun}\\
	% 	\sigma T_\text{gnd}^4 &= \begin{cases}
	% 		\sigma T_\text{gnd}^4C_\text{thin}N + F_\text{sun} & \text{optically thin}\\
	% 		\sigma T_\text{gnd}^4C_\text{thick} + F_\text{sun} & \text{optically thick}\end{cases}
	% \end{align*}
	% where $C_\text{thick/thin}$ are constants.
}

\ans{
	\centering
	\begin{align*}
		T_\text{gnd} &\propto \begin{cases}
			\sqrt[4]{e^{N\sigma_\text{scat}}} & \text{optically thin}\\
			\text{constant} & \text{optically thick}
			\end{cases}
	\end{align*}}

\sol{
	If the entire atmosphere is in bolometric (i.e. frequency-integrated or total-energy) equilibrium, then at every point in the atmosphere, the going-down flux of incoming energy must be balanced by the up-going flux of outgoing energy.
	By the planar symmetry of the problem, we know that there cannot be any dependence on angle or position (other than vertical distance above the plane).
	Antoehr way of saying this: For every component of flux that is not normal to the plane above some position, a nearby position will have a symmetrical flux that cancels it out.

	At the outermost layer of the atmosphere where there is no backscatter, we must have the incident flux from the sun, $F_\odot$, exactly balanced by the blackbody flux $\sigma_{SB}T^4$.
	If we are measuring the optical depth for an infrared photon to escape as $\tau = N\sigma$, then the $\tau = 0$ solution is
	\begin{align*}
		T &= \left(\frac{F_\odot}{\sigma_{SB}}\right)^\frac{1}{4}
	\end{align*}
	In the optically thin regime where $1 \gg \tau > 0$, we star to have some backscatter as infrared photons try to escape, and this backscattered light adds to the incident solar flux.
	Because we are so optically thin, $e^{-\tau} \approx 1-\tau$, so the flux leaving the outermost layer of the amosphere is attenuated by $1-\tau$ relative to the blackbody flux underneath.
	This means that $F_\odot = \sigma_{Sb}T^4(1-\tau)$, so
	\begin{align*}
		T &= \left(\frac{F_\odot}{\sigma_{SB}(1-N\sigma)}\right)^\frac{1}{4}
	\end{align*}
	This obviously perturbs the temperature slightly, but if forced to answer how $T$ scales with $N$, the fact that $N\sigma\ll 1$ means that it is effectively flat (the $N^0$ term dominates the Taylor expansion).

	In the optically thick regime, the exiting blackbody flux is attenuated by $e^{-\tau}$ and it really is in the exponential region this time: $\sigma_{SB}T^4 e^{-\tau} = F_\odot$, so
	\begin{align*}
		T &= \left(\frac{F_\odot e^{N\sigma}}{\sigma_{SB}}\right)^\frac{1}{4}
	\end{align*}
	This means that $T\propto e^\frac{N\sigma}{4}$.

	The point, in this simplified model, is that temperature goes from flat to exponential very rapidly with increasing column density as we cross the optically thin/thick boundary.
}