\qns{Blowing Str\"{o}mgren Bubbles}

\begin{itemize}
	\item Emitting $\eta\, \frac{\text{Lyman limit photons}}{\si{\second}}$
	\item Infinite hydrogen gas, number density $n$
\end{itemize}

\begin{enumerate}

\qitem{}
\work{
	\begin{align}
	    t_\text{rec} &= \frac{\lambda_\text{mfp,rec}}{v_\text{ion-e,rel}}\\
	        &\approx \frac{\lambda_\text{mfp,rec}}{v_e} \longleftarrow \text{roughly stationary ion}\\
	    \frac{1}{t_\text{rec}} &= \alpha n_e\\
	    	&= \langle v \sigma_\text{fb,e}(v)\rangle \cdot n_e\\
	        &= n_e \left\langle v\sigma_\text{bf}(f)\frac{g_0}{g_+}\left(\frac{hf}{m_ecv}\right)^2 \right\rangle \longleftarrow \text{Milne}\\
	        &\approx n_e \sigma_{\text{bf,$\chi$}} \frac{g_0}{g_+} \frac{\chi^2}{m_e^2c^2} \left\langle\frac{1}{v}\right\rangle \longleftarrow f = \frac{\chi}{h} \Rightarrow \sigma_\text{bf} = \sigma_{\text{bf},\chi} \approx 6(10^{-18})\si{\centi\meter^2}\\
	        &= n_e \sigma_\text{bf,$\chi$}\frac{g_0}{g_+}\frac{\chi^2}{m_e^2c^2}\sqrt{\frac{2\pi m_e}{k_BT}}\\
	        &\approx n\sigma_\text{bf,$\chi$}\frac{g_0}{g_+}\frac{\chi^2}{m_e^2c^2}\sqrt{\frac{2\pi m_e}{k_BT}} \longleftarrow n_e \approx n \text{ within the sphere (nearly fully ionized)}
	\end{align}
	\begin{itemize}
	    \item $\frac{g_0}{g_+} = 2$
	    \item $\chi = 13.6\si{\electronvolt}$
	    \item $\sigma_{\text{bf,}\chi} \approx 6\cdot 10^{-8}\si{\centi\meter^2}$
	\end{itemize}
	In other words, 
	$$t_\text{rec} = \frac{1}{n_e\alpha} \approx \frac{1}{n\alpha}$$
	where most everything above was just about solving for $\alpha$.}
\ans{
	\centering
	$$t_\text{rec} \approx \frac{1}{n\alpha} \approx 74,000\text{ years}$$}

\qitem{}
\work{
	The time it takes a photon at the edge of the bubble to encounter a neutral to ionize:
	\begin{align}
	    t_\text{ion} &= \frac{\lambda_\text{mfp,photon}}{c}\\
	        &= \frac{1}{cn_0\sigma_\text{bf,$\chi$}}\\
	        &\ll t_\text{rec}
	\end{align}
	Because the timescale of radiative recombination is long, we assume each photon ionizes an atom basically immediately after being emitted and that radiative recombination is negligible.
	That said, we're interested in how long it takes to emit enough photons to fill the sphere.
	\begin{align}
	    \eta t_\text{bubble} &= n\frac{4}{3}\pi r^3\\
	    t_\text{bubble} &= \frac{n\frac{4}{3}\pi r^3}{\eta}
	\end{align}
	where $r$ is the radius of the bubble, found by equating the photoionization rate to recombination within the sphere (after the bubble has formed)
	\begin{align}
	    \eta &= \frac{4}{3}\pi r^3 \cdot n_en_+\alpha\\
	    r &= \left(\frac{\eta}{\frac{4}{3}\pi n_en_+\alpha}\right)^\frac{1}{3}\\
	        &\approx \left(\frac{\eta}{\frac{4}{3}\pi n^2\alpha}\right)^\frac{1}{3}
	\end{align}
	which leaves us with 
	\begin{align}
	    t_\text{bubble} &\approx \frac{1}{n\alpha}
	\end{align}}
\ans{
	\centering
	$$ t_\text{bubble} \approx \frac{1}{n\alpha}$$}

\qitem{}
\work{
	Probably around one mean free path of a photon $\lambda_\text{mfp,photon} = \frac{1}{n_0\sigma_{\nu,\text{H}}} \approx \frac{1}{n_0\sigma_{\text{bf,}\chi}}$, where $n_0 = kn, k<1$ correction since a nontrivial portion of atoms are ionized}
\ans{
	\centering
	$$\ell_\text{boundary} \approx \lambda_\text{mfp,photon} \approx \frac{1}{kn\sigma_{\text{bf,}\chi}}$$
	$$k \approx 0.5$$
}
\end{enumerate}
