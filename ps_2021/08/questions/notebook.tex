
% Default to the notebook output style

    


% Inherit from the specified cell style.




    
\documentclass[11pt]{article}

    
    
    \usepackage[T1]{fontenc}
    % Nicer default font (+ math font) than Computer Modern for most use cases
    \usepackage{mathpazo}

    % Basic figure setup, for now with no caption control since it's done
    % automatically by Pandoc (which extracts ![](path) syntax from Markdown).
    \usepackage{graphicx}
    % We will generate all images so they have a width \maxwidth. This means
    % that they will get their normal width if they fit onto the page, but
    % are scaled down if they would overflow the margins.
    \makeatletter
    \def\maxwidth{\ifdim\Gin@nat@width>\linewidth\linewidth
    \else\Gin@nat@width\fi}
    \makeatother
    \let\Oldincludegraphics\includegraphics
    % Set max figure width to be 80% of text width, for now hardcoded.
    \renewcommand{\includegraphics}[1]{\Oldincludegraphics[width=.8\maxwidth]{#1}}
    % Ensure that by default, figures have no caption (until we provide a
    % proper Figure object with a Caption API and a way to capture that
    % in the conversion process - todo).
    \usepackage{caption}
    \DeclareCaptionLabelFormat{nolabel}{}
    \captionsetup{labelformat=nolabel}

    \usepackage{adjustbox} % Used to constrain images to a maximum size 
    \usepackage{xcolor} % Allow colors to be defined
    \usepackage{enumerate} % Needed for markdown enumerations to work
    \usepackage{geometry} % Used to adjust the document margins
    \usepackage{amsmath} % Equations
    \usepackage{amssymb} % Equations
    \usepackage{textcomp} % defines textquotesingle
    % Hack from http://tex.stackexchange.com/a/47451/13684:
    \AtBeginDocument{%
        \def\PYZsq{\textquotesingle}% Upright quotes in Pygmentized code
    }
    \usepackage{upquote} % Upright quotes for verbatim code
    \usepackage{eurosym} % defines \euro
    \usepackage[mathletters]{ucs} % Extended unicode (utf-8) support
    \usepackage[utf8x]{inputenc} % Allow utf-8 characters in the tex document
    \usepackage{fancyvrb} % verbatim replacement that allows latex
    \usepackage{grffile} % extends the file name processing of package graphics 
                         % to support a larger range 
    % The hyperref package gives us a pdf with properly built
    % internal navigation ('pdf bookmarks' for the table of contents,
    % internal cross-reference links, web links for URLs, etc.)
    \usepackage{hyperref}
    \usepackage{longtable} % longtable support required by pandoc >1.10
    \usepackage{booktabs}  % table support for pandoc > 1.12.2
    \usepackage[inline]{enumitem} % IRkernel/repr support (it uses the enumerate* environment)
    \usepackage[normalem]{ulem} % ulem is needed to support strikethroughs (\sout)
                                % normalem makes italics be italics, not underlines
    

    
    
    % Colors for the hyperref package
    \definecolor{urlcolor}{rgb}{0,.145,.698}
    \definecolor{linkcolor}{rgb}{.71,0.21,0.01}
    \definecolor{citecolor}{rgb}{.12,.54,.11}

    % ANSI colors
    \definecolor{ansi-black}{HTML}{3E424D}
    \definecolor{ansi-black-intense}{HTML}{282C36}
    \definecolor{ansi-red}{HTML}{E75C58}
    \definecolor{ansi-red-intense}{HTML}{B22B31}
    \definecolor{ansi-green}{HTML}{00A250}
    \definecolor{ansi-green-intense}{HTML}{007427}
    \definecolor{ansi-yellow}{HTML}{DDB62B}
    \definecolor{ansi-yellow-intense}{HTML}{B27D12}
    \definecolor{ansi-blue}{HTML}{208FFB}
    \definecolor{ansi-blue-intense}{HTML}{0065CA}
    \definecolor{ansi-magenta}{HTML}{D160C4}
    \definecolor{ansi-magenta-intense}{HTML}{A03196}
    \definecolor{ansi-cyan}{HTML}{60C6C8}
    \definecolor{ansi-cyan-intense}{HTML}{258F8F}
    \definecolor{ansi-white}{HTML}{C5C1B4}
    \definecolor{ansi-white-intense}{HTML}{A1A6B2}

    % commands and environments needed by pandoc snippets
    % extracted from the output of `pandoc -s`
    \providecommand{\tightlist}{%
      \setlength{\itemsep}{0pt}\setlength{\parskip}{0pt}}
    \DefineVerbatimEnvironment{Highlighting}{Verbatim}{commandchars=\\\{\}}
    % Add ',fontsize=\small' for more characters per line
    \newenvironment{Shaded}{}{}
    \newcommand{\KeywordTok}[1]{\textcolor[rgb]{0.00,0.44,0.13}{\textbf{{#1}}}}
    \newcommand{\DataTypeTok}[1]{\textcolor[rgb]{0.56,0.13,0.00}{{#1}}}
    \newcommand{\DecValTok}[1]{\textcolor[rgb]{0.25,0.63,0.44}{{#1}}}
    \newcommand{\BaseNTok}[1]{\textcolor[rgb]{0.25,0.63,0.44}{{#1}}}
    \newcommand{\FloatTok}[1]{\textcolor[rgb]{0.25,0.63,0.44}{{#1}}}
    \newcommand{\CharTok}[1]{\textcolor[rgb]{0.25,0.44,0.63}{{#1}}}
    \newcommand{\StringTok}[1]{\textcolor[rgb]{0.25,0.44,0.63}{{#1}}}
    \newcommand{\CommentTok}[1]{\textcolor[rgb]{0.38,0.63,0.69}{\textit{{#1}}}}
    \newcommand{\OtherTok}[1]{\textcolor[rgb]{0.00,0.44,0.13}{{#1}}}
    \newcommand{\AlertTok}[1]{\textcolor[rgb]{1.00,0.00,0.00}{\textbf{{#1}}}}
    \newcommand{\FunctionTok}[1]{\textcolor[rgb]{0.02,0.16,0.49}{{#1}}}
    \newcommand{\RegionMarkerTok}[1]{{#1}}
    \newcommand{\ErrorTok}[1]{\textcolor[rgb]{1.00,0.00,0.00}{\textbf{{#1}}}}
    \newcommand{\NormalTok}[1]{{#1}}
    
    % Additional commands for more recent versions of Pandoc
    \newcommand{\ConstantTok}[1]{\textcolor[rgb]{0.53,0.00,0.00}{{#1}}}
    \newcommand{\SpecialCharTok}[1]{\textcolor[rgb]{0.25,0.44,0.63}{{#1}}}
    \newcommand{\VerbatimStringTok}[1]{\textcolor[rgb]{0.25,0.44,0.63}{{#1}}}
    \newcommand{\SpecialStringTok}[1]{\textcolor[rgb]{0.73,0.40,0.53}{{#1}}}
    \newcommand{\ImportTok}[1]{{#1}}
    \newcommand{\DocumentationTok}[1]{\textcolor[rgb]{0.73,0.13,0.13}{\textit{{#1}}}}
    \newcommand{\AnnotationTok}[1]{\textcolor[rgb]{0.38,0.63,0.69}{\textbf{\textit{{#1}}}}}
    \newcommand{\CommentVarTok}[1]{\textcolor[rgb]{0.38,0.63,0.69}{\textbf{\textit{{#1}}}}}
    \newcommand{\VariableTok}[1]{\textcolor[rgb]{0.10,0.09,0.49}{{#1}}}
    \newcommand{\ControlFlowTok}[1]{\textcolor[rgb]{0.00,0.44,0.13}{\textbf{{#1}}}}
    \newcommand{\OperatorTok}[1]{\textcolor[rgb]{0.40,0.40,0.40}{{#1}}}
    \newcommand{\BuiltInTok}[1]{{#1}}
    \newcommand{\ExtensionTok}[1]{{#1}}
    \newcommand{\PreprocessorTok}[1]{\textcolor[rgb]{0.74,0.48,0.00}{{#1}}}
    \newcommand{\AttributeTok}[1]{\textcolor[rgb]{0.49,0.56,0.16}{{#1}}}
    \newcommand{\InformationTok}[1]{\textcolor[rgb]{0.38,0.63,0.69}{\textbf{\textit{{#1}}}}}
    \newcommand{\WarningTok}[1]{\textcolor[rgb]{0.38,0.63,0.69}{\textbf{\textit{{#1}}}}}
    
    
    % Define a nice break command that doesn't care if a line doesn't already
    % exist.
    \def\br{\hspace*{\fill} \\* }
    % Math Jax compatability definitions
    \def\gt{>}
    \def\lt{<}
    % Document parameters
    \title{q\_pulsar\_dispersion\_measure\_ipynb}
    
    
    

    % Pygments definitions
    
\makeatletter
\def\PY@reset{\let\PY@it=\relax \let\PY@bf=\relax%
    \let\PY@ul=\relax \let\PY@tc=\relax%
    \let\PY@bc=\relax \let\PY@ff=\relax}
\def\PY@tok#1{\csname PY@tok@#1\endcsname}
\def\PY@toks#1+{\ifx\relax#1\empty\else%
    \PY@tok{#1}\expandafter\PY@toks\fi}
\def\PY@do#1{\PY@bc{\PY@tc{\PY@ul{%
    \PY@it{\PY@bf{\PY@ff{#1}}}}}}}
\def\PY#1#2{\PY@reset\PY@toks#1+\relax+\PY@do{#2}}

\expandafter\def\csname PY@tok@w\endcsname{\def\PY@tc##1{\textcolor[rgb]{0.73,0.73,0.73}{##1}}}
\expandafter\def\csname PY@tok@c\endcsname{\let\PY@it=\textit\def\PY@tc##1{\textcolor[rgb]{0.25,0.50,0.50}{##1}}}
\expandafter\def\csname PY@tok@cp\endcsname{\def\PY@tc##1{\textcolor[rgb]{0.74,0.48,0.00}{##1}}}
\expandafter\def\csname PY@tok@k\endcsname{\let\PY@bf=\textbf\def\PY@tc##1{\textcolor[rgb]{0.00,0.50,0.00}{##1}}}
\expandafter\def\csname PY@tok@kp\endcsname{\def\PY@tc##1{\textcolor[rgb]{0.00,0.50,0.00}{##1}}}
\expandafter\def\csname PY@tok@kt\endcsname{\def\PY@tc##1{\textcolor[rgb]{0.69,0.00,0.25}{##1}}}
\expandafter\def\csname PY@tok@o\endcsname{\def\PY@tc##1{\textcolor[rgb]{0.40,0.40,0.40}{##1}}}
\expandafter\def\csname PY@tok@ow\endcsname{\let\PY@bf=\textbf\def\PY@tc##1{\textcolor[rgb]{0.67,0.13,1.00}{##1}}}
\expandafter\def\csname PY@tok@nb\endcsname{\def\PY@tc##1{\textcolor[rgb]{0.00,0.50,0.00}{##1}}}
\expandafter\def\csname PY@tok@nf\endcsname{\def\PY@tc##1{\textcolor[rgb]{0.00,0.00,1.00}{##1}}}
\expandafter\def\csname PY@tok@nc\endcsname{\let\PY@bf=\textbf\def\PY@tc##1{\textcolor[rgb]{0.00,0.00,1.00}{##1}}}
\expandafter\def\csname PY@tok@nn\endcsname{\let\PY@bf=\textbf\def\PY@tc##1{\textcolor[rgb]{0.00,0.00,1.00}{##1}}}
\expandafter\def\csname PY@tok@ne\endcsname{\let\PY@bf=\textbf\def\PY@tc##1{\textcolor[rgb]{0.82,0.25,0.23}{##1}}}
\expandafter\def\csname PY@tok@nv\endcsname{\def\PY@tc##1{\textcolor[rgb]{0.10,0.09,0.49}{##1}}}
\expandafter\def\csname PY@tok@no\endcsname{\def\PY@tc##1{\textcolor[rgb]{0.53,0.00,0.00}{##1}}}
\expandafter\def\csname PY@tok@nl\endcsname{\def\PY@tc##1{\textcolor[rgb]{0.63,0.63,0.00}{##1}}}
\expandafter\def\csname PY@tok@ni\endcsname{\let\PY@bf=\textbf\def\PY@tc##1{\textcolor[rgb]{0.60,0.60,0.60}{##1}}}
\expandafter\def\csname PY@tok@na\endcsname{\def\PY@tc##1{\textcolor[rgb]{0.49,0.56,0.16}{##1}}}
\expandafter\def\csname PY@tok@nt\endcsname{\let\PY@bf=\textbf\def\PY@tc##1{\textcolor[rgb]{0.00,0.50,0.00}{##1}}}
\expandafter\def\csname PY@tok@nd\endcsname{\def\PY@tc##1{\textcolor[rgb]{0.67,0.13,1.00}{##1}}}
\expandafter\def\csname PY@tok@s\endcsname{\def\PY@tc##1{\textcolor[rgb]{0.73,0.13,0.13}{##1}}}
\expandafter\def\csname PY@tok@sd\endcsname{\let\PY@it=\textit\def\PY@tc##1{\textcolor[rgb]{0.73,0.13,0.13}{##1}}}
\expandafter\def\csname PY@tok@si\endcsname{\let\PY@bf=\textbf\def\PY@tc##1{\textcolor[rgb]{0.73,0.40,0.53}{##1}}}
\expandafter\def\csname PY@tok@se\endcsname{\let\PY@bf=\textbf\def\PY@tc##1{\textcolor[rgb]{0.73,0.40,0.13}{##1}}}
\expandafter\def\csname PY@tok@sr\endcsname{\def\PY@tc##1{\textcolor[rgb]{0.73,0.40,0.53}{##1}}}
\expandafter\def\csname PY@tok@ss\endcsname{\def\PY@tc##1{\textcolor[rgb]{0.10,0.09,0.49}{##1}}}
\expandafter\def\csname PY@tok@sx\endcsname{\def\PY@tc##1{\textcolor[rgb]{0.00,0.50,0.00}{##1}}}
\expandafter\def\csname PY@tok@m\endcsname{\def\PY@tc##1{\textcolor[rgb]{0.40,0.40,0.40}{##1}}}
\expandafter\def\csname PY@tok@gh\endcsname{\let\PY@bf=\textbf\def\PY@tc##1{\textcolor[rgb]{0.00,0.00,0.50}{##1}}}
\expandafter\def\csname PY@tok@gu\endcsname{\let\PY@bf=\textbf\def\PY@tc##1{\textcolor[rgb]{0.50,0.00,0.50}{##1}}}
\expandafter\def\csname PY@tok@gd\endcsname{\def\PY@tc##1{\textcolor[rgb]{0.63,0.00,0.00}{##1}}}
\expandafter\def\csname PY@tok@gi\endcsname{\def\PY@tc##1{\textcolor[rgb]{0.00,0.63,0.00}{##1}}}
\expandafter\def\csname PY@tok@gr\endcsname{\def\PY@tc##1{\textcolor[rgb]{1.00,0.00,0.00}{##1}}}
\expandafter\def\csname PY@tok@ge\endcsname{\let\PY@it=\textit}
\expandafter\def\csname PY@tok@gs\endcsname{\let\PY@bf=\textbf}
\expandafter\def\csname PY@tok@gp\endcsname{\let\PY@bf=\textbf\def\PY@tc##1{\textcolor[rgb]{0.00,0.00,0.50}{##1}}}
\expandafter\def\csname PY@tok@go\endcsname{\def\PY@tc##1{\textcolor[rgb]{0.53,0.53,0.53}{##1}}}
\expandafter\def\csname PY@tok@gt\endcsname{\def\PY@tc##1{\textcolor[rgb]{0.00,0.27,0.87}{##1}}}
\expandafter\def\csname PY@tok@err\endcsname{\def\PY@bc##1{\setlength{\fboxsep}{0pt}\fcolorbox[rgb]{1.00,0.00,0.00}{1,1,1}{\strut ##1}}}
\expandafter\def\csname PY@tok@kc\endcsname{\let\PY@bf=\textbf\def\PY@tc##1{\textcolor[rgb]{0.00,0.50,0.00}{##1}}}
\expandafter\def\csname PY@tok@kd\endcsname{\let\PY@bf=\textbf\def\PY@tc##1{\textcolor[rgb]{0.00,0.50,0.00}{##1}}}
\expandafter\def\csname PY@tok@kn\endcsname{\let\PY@bf=\textbf\def\PY@tc##1{\textcolor[rgb]{0.00,0.50,0.00}{##1}}}
\expandafter\def\csname PY@tok@kr\endcsname{\let\PY@bf=\textbf\def\PY@tc##1{\textcolor[rgb]{0.00,0.50,0.00}{##1}}}
\expandafter\def\csname PY@tok@bp\endcsname{\def\PY@tc##1{\textcolor[rgb]{0.00,0.50,0.00}{##1}}}
\expandafter\def\csname PY@tok@fm\endcsname{\def\PY@tc##1{\textcolor[rgb]{0.00,0.00,1.00}{##1}}}
\expandafter\def\csname PY@tok@vc\endcsname{\def\PY@tc##1{\textcolor[rgb]{0.10,0.09,0.49}{##1}}}
\expandafter\def\csname PY@tok@vg\endcsname{\def\PY@tc##1{\textcolor[rgb]{0.10,0.09,0.49}{##1}}}
\expandafter\def\csname PY@tok@vi\endcsname{\def\PY@tc##1{\textcolor[rgb]{0.10,0.09,0.49}{##1}}}
\expandafter\def\csname PY@tok@vm\endcsname{\def\PY@tc##1{\textcolor[rgb]{0.10,0.09,0.49}{##1}}}
\expandafter\def\csname PY@tok@sa\endcsname{\def\PY@tc##1{\textcolor[rgb]{0.73,0.13,0.13}{##1}}}
\expandafter\def\csname PY@tok@sb\endcsname{\def\PY@tc##1{\textcolor[rgb]{0.73,0.13,0.13}{##1}}}
\expandafter\def\csname PY@tok@sc\endcsname{\def\PY@tc##1{\textcolor[rgb]{0.73,0.13,0.13}{##1}}}
\expandafter\def\csname PY@tok@dl\endcsname{\def\PY@tc##1{\textcolor[rgb]{0.73,0.13,0.13}{##1}}}
\expandafter\def\csname PY@tok@s2\endcsname{\def\PY@tc##1{\textcolor[rgb]{0.73,0.13,0.13}{##1}}}
\expandafter\def\csname PY@tok@sh\endcsname{\def\PY@tc##1{\textcolor[rgb]{0.73,0.13,0.13}{##1}}}
\expandafter\def\csname PY@tok@s1\endcsname{\def\PY@tc##1{\textcolor[rgb]{0.73,0.13,0.13}{##1}}}
\expandafter\def\csname PY@tok@mb\endcsname{\def\PY@tc##1{\textcolor[rgb]{0.40,0.40,0.40}{##1}}}
\expandafter\def\csname PY@tok@mf\endcsname{\def\PY@tc##1{\textcolor[rgb]{0.40,0.40,0.40}{##1}}}
\expandafter\def\csname PY@tok@mh\endcsname{\def\PY@tc##1{\textcolor[rgb]{0.40,0.40,0.40}{##1}}}
\expandafter\def\csname PY@tok@mi\endcsname{\def\PY@tc##1{\textcolor[rgb]{0.40,0.40,0.40}{##1}}}
\expandafter\def\csname PY@tok@il\endcsname{\def\PY@tc##1{\textcolor[rgb]{0.40,0.40,0.40}{##1}}}
\expandafter\def\csname PY@tok@mo\endcsname{\def\PY@tc##1{\textcolor[rgb]{0.40,0.40,0.40}{##1}}}
\expandafter\def\csname PY@tok@ch\endcsname{\let\PY@it=\textit\def\PY@tc##1{\textcolor[rgb]{0.25,0.50,0.50}{##1}}}
\expandafter\def\csname PY@tok@cm\endcsname{\let\PY@it=\textit\def\PY@tc##1{\textcolor[rgb]{0.25,0.50,0.50}{##1}}}
\expandafter\def\csname PY@tok@cpf\endcsname{\let\PY@it=\textit\def\PY@tc##1{\textcolor[rgb]{0.25,0.50,0.50}{##1}}}
\expandafter\def\csname PY@tok@c1\endcsname{\let\PY@it=\textit\def\PY@tc##1{\textcolor[rgb]{0.25,0.50,0.50}{##1}}}
\expandafter\def\csname PY@tok@cs\endcsname{\let\PY@it=\textit\def\PY@tc##1{\textcolor[rgb]{0.25,0.50,0.50}{##1}}}

\def\PYZbs{\char`\\}
\def\PYZus{\char`\_}
\def\PYZob{\char`\{}
\def\PYZcb{\char`\}}
\def\PYZca{\char`\^}
\def\PYZam{\char`\&}
\def\PYZlt{\char`\<}
\def\PYZgt{\char`\>}
\def\PYZsh{\char`\#}
\def\PYZpc{\char`\%}
\def\PYZdl{\char`\$}
\def\PYZhy{\char`\-}
\def\PYZsq{\char`\'}
\def\PYZdq{\char`\"}
\def\PYZti{\char`\~}
% for compatibility with earlier versions
\def\PYZat{@}
\def\PYZlb{[}
\def\PYZrb{]}
\makeatother


    % Exact colors from NB
    \definecolor{incolor}{rgb}{0.0, 0.0, 0.5}
    \definecolor{outcolor}{rgb}{0.545, 0.0, 0.0}



    
    % Prevent overflowing lines due to hard-to-break entities
    \sloppy 
    % Setup hyperref package
    \hypersetup{
      breaklinks=true,  % so long urls are correctly broken across lines
      colorlinks=true,
      urlcolor=urlcolor,
      linkcolor=linkcolor,
      citecolor=citecolor,
      }
    % Slightly bigger margins than the latex defaults
    
    \geometry{verbose,tmargin=1in,bmargin=1in,lmargin=1in,rmargin=1in}
    
    

    \begin{document}
    
    
    \maketitle
    
    

    
    \begin{Verbatim}[commandchars=\\\{\}]
{\color{incolor}In [{\color{incolor}1}]:} \PY{o}{\PYZpc{}}\PY{k}{matplotlib} inline
        \PY{k+kn}{import} \PY{n+nn}{numpy} \PY{k}{as} \PY{n+nn}{np}
        \PY{k+kn}{import} \PY{n+nn}{matplotlib}\PY{n+nn}{.}\PY{n+nn}{pyplot} \PY{k}{as} \PY{n+nn}{plt}
        \PY{k+kn}{from} \PY{n+nn}{matplotlib} \PY{k}{import} \PY{n}{cm}
        \PY{k+kn}{from} \PY{n+nn}{mpl\PYZus{}toolkits}\PY{n+nn}{.}\PY{n+nn}{mplot3d} \PY{k}{import} \PY{n}{Axes3D}
        \PY{k+kn}{from} \PY{n+nn}{scipy}\PY{n+nn}{.}\PY{n+nn}{stats}\PY{n+nn}{.}\PY{n+nn}{mstats} \PY{k}{import} \PY{n}{gmean}
        \PY{k+kn}{import} \PY{n+nn}{time}
        \PY{k+kn}{import} \PY{n+nn}{pylab} \PY{k}{as} \PY{n+nn}{pl}
        \PY{k+kn}{from} \PY{n+nn}{IPython} \PY{k}{import} \PY{n}{display}
        
        \PY{k+kn}{from} \PY{n+nn}{pprint} \PY{k}{import} \PY{n}{pprint}
\end{Verbatim}


    \begin{Verbatim}[commandchars=\\\{\}]
{\color{incolor}In [{\color{incolor}2}]:} \PY{n}{float\PYZus{}formatter} \PY{o}{=} \PY{l+s+s2}{\PYZdq{}}\PY{l+s+si}{\PYZob{}0:0.2e\PYZcb{}}\PY{l+s+s2}{\PYZdq{}}\PY{o}{.}\PY{n}{format}
        \PY{n}{np}\PY{o}{.}\PY{n}{set\PYZus{}printoptions}\PY{p}{(}\PY{n}{formatter}\PY{o}{=}\PY{p}{\PYZob{}}\PY{l+s+s1}{\PYZsq{}}\PY{l+s+s1}{float\PYZus{}kind}\PY{l+s+s1}{\PYZsq{}}\PY{p}{:}\PY{n}{float\PYZus{}formatter}\PY{p}{\PYZcb{}}\PY{p}{)}
\end{Verbatim}


    \begin{Verbatim}[commandchars=\\\{\}]
{\color{incolor}In [{\color{incolor}3}]:} \PY{n}{q} \PY{o}{=} \PY{l+m+mf}{5e\PYZhy{}10} \PY{c+c1}{\PYZsh{} esu, electron charge}
        \PY{n}{m\PYZus{}e} \PY{o}{=} \PY{l+m+mf}{1e\PYZhy{}27} \PY{c+c1}{\PYZsh{} g, electron mass}
        \PY{n}{c} \PY{o}{=} \PY{l+m+mf}{3e10} \PY{c+c1}{\PYZsh{} cm/s, speed of light}
        \PY{n}{cm2pc} \PY{o}{=} \PY{l+m+mf}{3.24077929e\PYZhy{}19}
        
        \PY{n}{prefactor} \PY{o}{=} \PY{n}{q}\PY{o}{*}\PY{o}{*}\PY{l+m+mi}{2} \PY{o}{/} \PY{p}{(}\PY{l+m+mi}{2}\PY{o}{*}\PY{n}{np}\PY{o}{.}\PY{n}{pi} \PY{o}{*} \PY{n}{m\PYZus{}e} \PY{o}{*} \PY{n}{c}\PY{p}{)}
        
        \PY{k}{def} \PY{n+nf}{estimate\PYZus{}DM}\PY{p}{(}\PY{n}{nu0}\PY{p}{,} \PY{n}{nu1}\PY{p}{,} \PY{n}{t0}\PY{p}{,} \PY{n}{t1}\PY{p}{)}\PY{p}{:}
            \PY{l+s+sd}{\PYZsq{}\PYZsq{}\PYZsq{}}
        \PY{l+s+sd}{    Inputs:}
        \PY{l+s+sd}{        nu0/1: Float. Frequencies (in Hz) of peaks.}
        \PY{l+s+sd}{        t0/1: Float. Time of arrival of the peaks.}
        \PY{l+s+sd}{    Returns:}
        \PY{l+s+sd}{        DM: An estimate of the dispersion measure (i.e. column density of electrons)}
        \PY{l+s+sd}{        in electrons/cm\PYZca{}2.}
        \PY{l+s+sd}{    \PYZsq{}\PYZsq{}\PYZsq{}}
        \PY{c+c1}{\PYZsh{}     prefactor = 4140 * 1e6**2 * cm2pc}
            \PY{n}{prefactor} \PY{o}{=} \PY{n}{q}\PY{o}{*}\PY{o}{*}\PY{l+m+mi}{2} \PY{o}{/} \PY{p}{(}\PY{l+m+mi}{2}\PY{o}{*}\PY{n}{np}\PY{o}{.}\PY{n}{pi} \PY{o}{*} \PY{n}{m\PYZus{}e} \PY{o}{*} \PY{n}{c}\PY{p}{)}
            \PY{n}{DM} \PY{o}{=} \PY{p}{(}\PY{n}{t1}\PY{o}{\PYZhy{}}\PY{n}{t0}\PY{p}{)} \PY{o}{/} \PY{p}{(}\PY{n}{prefactor} \PY{o}{*} \PY{p}{(}\PY{l+m+mi}{1}\PY{o}{/}\PY{n}{nu1}\PY{o}{*}\PY{o}{*}\PY{l+m+mi}{2} \PY{o}{\PYZhy{}} \PY{l+m+mi}{1}\PY{o}{/}\PY{n}{nu0}\PY{o}{*}\PY{o}{*}\PY{l+m+mi}{2}\PY{p}{)}\PY{p}{)}
            \PY{k}{return} \PY{n}{DM}
\end{Verbatim}


    2

2.1

    \begin{Verbatim}[commandchars=\\\{\}]
{\color{incolor}In [{\color{incolor}4}]:} \PY{n}{fname\PYZus{}pulsar} \PY{o}{=} \PY{l+s+s1}{\PYZsq{}}\PY{l+s+s1}{./pulsar.dat}\PY{l+s+s1}{\PYZsq{}}
        
        \PY{c+c1}{\PYZsh{} Getting frequencies}
        \PY{k}{with} \PY{n+nb}{open}\PY{p}{(}\PY{n}{fname\PYZus{}pulsar}\PY{p}{)} \PY{k}{as} \PY{n}{f}\PY{p}{:}
            \PY{n}{column\PYZus{}headers} \PY{o}{=} \PY{n}{f}\PY{o}{.}\PY{n}{readlines}\PY{p}{(}\PY{p}{)}\PY{p}{[}\PY{l+m+mi}{0}\PY{p}{]}\PY{o}{.}\PY{n}{strip}\PY{p}{(}\PY{p}{)}\PY{o}{.}\PY{n}{split}\PY{p}{(}\PY{l+s+s1}{\PYZsq{}}\PY{l+s+s1}{] [}\PY{l+s+s1}{\PYZsq{}}\PY{p}{)}
        \PY{n}{nu\PYZus{}vec} \PY{o}{=} \PY{n}{np}\PY{o}{.}\PY{n}{array}\PY{p}{(}\PY{p}{[}\PY{l+m+mf}{1e9}\PY{o}{*}\PY{n+nb}{float}\PY{p}{(}\PY{n}{nu\PYZus{}str}\PY{o}{.}\PY{n}{replace}\PY{p}{(}\PY{l+s+s1}{\PYZsq{}}\PY{l+s+s1}{GHz}\PY{l+s+s1}{\PYZsq{}}\PY{p}{,} \PY{l+s+s1}{\PYZsq{}}\PY{l+s+s1}{\PYZsq{}}\PY{p}{)}\PY{o}{.}\PY{n}{replace}\PY{p}{(}\PY{l+s+s1}{\PYZsq{}}\PY{l+s+s1}{]}\PY{l+s+s1}{\PYZsq{}}\PY{p}{,} \PY{l+s+s1}{\PYZsq{}}\PY{l+s+s1}{\PYZsq{}}\PY{p}{)}\PY{p}{)} \PY{k}{for} \PY{n}{nu\PYZus{}str} \PY{o+ow}{in} \PY{n}{column\PYZus{}headers}\PY{p}{[}\PY{l+m+mi}{1}\PY{p}{:}\PY{p}{]}\PY{p}{]}\PY{p}{)}
        
        \PY{c+c1}{\PYZsh{} Data}
        \PY{n}{data\PYZus{}raw} \PY{o}{=} \PY{n}{np}\PY{o}{.}\PY{n}{loadtxt}\PY{p}{(}\PY{n}{fname\PYZus{}pulsar}\PY{p}{)}
        
        \PY{n}{t\PYZus{}vec} \PY{o}{=} \PY{n}{data\PYZus{}raw}\PY{p}{[}\PY{p}{:}\PY{p}{,}\PY{l+m+mi}{0}\PY{p}{]}
        \PY{n}{t\PYZus{}mesh}\PY{p}{,} \PY{n}{nu\PYZus{}mesh} \PY{o}{=} \PY{n}{np}\PY{o}{.}\PY{n}{meshgrid}\PY{p}{(}\PY{n}{t\PYZus{}vec}\PY{p}{,} \PY{n}{nu\PYZus{}vec}\PY{o}{/}\PY{l+m+mf}{1e9}\PY{p}{)}
        \PY{n}{data\PYZus{}pulsar} \PY{o}{=} \PY{n}{data\PYZus{}raw}\PY{p}{[}\PY{p}{:}\PY{p}{,}\PY{l+m+mi}{1}\PY{p}{:}\PY{p}{]}\PY{o}{.}\PY{n}{transpose}\PY{p}{(}\PY{p}{)}
        
        \PY{c+c1}{\PYZsh{} Plot the surface}
        \PY{n}{fig}\PY{p}{,} \PY{n}{ax} \PY{o}{=} \PY{n}{plt}\PY{o}{.}\PY{n}{subplots}\PY{p}{(}\PY{n}{subplot\PYZus{}kw}\PY{o}{=}\PY{p}{\PYZob{}}\PY{l+s+s2}{\PYZdq{}}\PY{l+s+s2}{projection}\PY{l+s+s2}{\PYZdq{}}\PY{p}{:} \PY{l+s+s2}{\PYZdq{}}\PY{l+s+s2}{3d}\PY{l+s+s2}{\PYZdq{}}\PY{p}{\PYZcb{}}\PY{p}{)}
        \PY{n}{surf} \PY{o}{=} \PY{n}{ax}\PY{o}{.}\PY{n}{plot\PYZus{}surface}\PY{p}{(}\PY{n}{t\PYZus{}mesh}\PY{p}{,} \PY{n}{nu\PYZus{}mesh}\PY{p}{,} \PY{n}{data\PYZus{}pulsar}\PY{p}{,} \PY{n}{linewidth}\PY{o}{=}\PY{l+m+mi}{0}\PY{p}{,} \PY{n}{antialiased}\PY{o}{=}\PY{k+kc}{False}\PY{p}{)}
        \PY{n}{ax}\PY{o}{.}\PY{n}{set\PYZus{}xlabel}\PY{p}{(}\PY{l+s+s1}{\PYZsq{}}\PY{l+s+s1}{Time [s]}\PY{l+s+s1}{\PYZsq{}}\PY{p}{)}
        \PY{n}{ax}\PY{o}{.}\PY{n}{set\PYZus{}ylabel}\PY{p}{(}\PY{l+s+s1}{\PYZsq{}}\PY{l+s+s1}{Frequency [GHz]}\PY{l+s+s1}{\PYZsq{}}\PY{p}{)}
        \PY{n}{ax}\PY{o}{.}\PY{n}{set\PYZus{}zlabel}\PY{p}{(}\PY{l+s+s1}{\PYZsq{}}\PY{l+s+s1}{Power [AU]}\PY{l+s+s1}{\PYZsq{}}\PY{p}{)}
        
        \PY{c+c1}{\PYZsh{} for i in range(0, 100, 10):}
        \PY{c+c1}{\PYZsh{}     plt.figure()}
        \PY{c+c1}{\PYZsh{}     plt.plot(t\PYZus{}vec, data\PYZus{}pulsar[i])}
        \PY{c+c1}{\PYZsh{}     plt.title(f\PYZsq{}\PYZob{}nu\PYZus{}vec[i]\PYZcb{} GHz\PYZsq{})}
        
        \PY{c+c1}{\PYZsh{} Data too noisy for cross\PYZhy{}correlations of raw data}
        \PY{c+c1}{\PYZsh{} for d in data\PYZus{}pulsar[11:20]:}
        \PY{c+c1}{\PYZsh{}     plt.figure()}
        \PY{c+c1}{\PYZsh{}     cross\PYZus{}corr = np.convolve(data\PYZus{}pulsar[0], d[::\PYZhy{}1], mode=\PYZsq{}same\PYZsq{})}
        \PY{c+c1}{\PYZsh{}     cross\PYZus{}corr\PYZus{}norm = cross\PYZus{}corr/max(cross\PYZus{}corr)}
        \PY{c+c1}{\PYZsh{}     plt.plot(t\PYZus{}vec, cross\PYZus{}corr\PYZus{}norm)}
        
        \PY{c+c1}{\PYZsh{} print(len(column\PYZus{}headers))}
        \PY{c+c1}{\PYZsh{} print(len(set(column\PYZus{}headers)))}
        \PY{c+c1}{\PYZsh{} print(column\PYZus{}headers.index(\PYZsq{}1.022 GHz\PYZsq{}))}
        \PY{c+c1}{\PYZsh{} print(column\PYZus{}headers[23])}
        \PY{c+c1}{\PYZsh{} print(column\PYZus{}headers[24])}
\end{Verbatim}


\begin{Verbatim}[commandchars=\\\{\}]
{\color{outcolor}Out[{\color{outcolor}4}]:} Text(0.5,0,'Power [AU]')
\end{Verbatim}
            
    \begin{center}
    \adjustimage{max size={0.9\linewidth}{0.9\paperheight}}{output_4_1.png}
    \end{center}
    { \hspace*{\fill} \\}
    
    \begin{Verbatim}[commandchars=\\\{\}]
{\color{incolor}In [{\color{incolor}5}]:} \PY{c+c1}{\PYZsh{} Indices of top 2 peaks for a given frequency}
        \PY{n}{idx\PYZus{}pk\PYZus{}dict} \PY{o}{=} \PY{n+nb}{dict}\PY{p}{(}\PY{p}{)}
        \PY{k}{for} \PY{n}{i}\PY{p}{,}\PY{n}{nu} \PY{o+ow}{in} \PY{n+nb}{enumerate}\PY{p}{(}\PY{n}{nu\PYZus{}vec}\PY{p}{)}\PY{p}{:}
            \PY{n}{idx\PYZus{}pk} \PY{o}{=} \PY{n}{np}\PY{o}{.}\PY{n}{argsort}\PY{p}{(}\PY{n}{data\PYZus{}pulsar}\PY{p}{[}\PY{n}{i}\PY{p}{]}\PY{p}{)}\PY{p}{[}\PY{p}{:}\PY{p}{:}\PY{o}{\PYZhy{}}\PY{l+m+mi}{1}\PY{p}{]}\PY{p}{[}\PY{l+m+mi}{0}\PY{p}{:}\PY{l+m+mi}{2}\PY{p}{]}
            \PY{n}{idx\PYZus{}pk\PYZus{}dict}\PY{p}{[}\PY{n}{i}\PY{p}{]} \PY{o}{=} \PY{n}{idx\PYZus{}pk}
            
        \PY{c+c1}{\PYZsh{} Filter to get (indices of) frequencies which have a single \PYZdq{}clearly\PYZdq{} distinguishable pulse}
        \PY{c+c1}{\PYZsh{} (defined by a minimum difference in two max peaks\PYZsq{} respective heights)}
        \PY{c+c1}{\PYZsh{} The keys are the indices matching up to nu\PYZus{}vec.}
        
        \PY{c+c1}{\PYZsh{} The threshold for what constitutes a single clearly distinguishable peak}
        \PY{n}{diff\PYZus{}threshold\PYZus{}dict} \PY{o}{=} \PY{p}{\PYZob{}}\PY{n}{i}\PY{p}{:}\PY{n}{np}\PY{o}{.}\PY{n}{max}\PY{p}{(}\PY{n}{data\PYZus{}pulsar}\PY{p}{[}\PY{n}{i}\PY{p}{]}\PY{p}{)}\PY{o}{*}\PY{l+m+mf}{0.6} \PY{k}{for} \PY{n}{i}\PY{p}{,}\PY{n}{\PYZus{}} \PY{o+ow}{in} \PY{n+nb}{enumerate}\PY{p}{(}\PY{n}{nu\PYZus{}vec}\PY{p}{)}\PY{p}{\PYZcb{}}
        
        \PY{c+c1}{\PYZsh{} The difference between peaks at a given time}
        \PY{n}{diff\PYZus{}pk\PYZus{}dict} \PY{o}{=} \PY{p}{\PYZob{}}\PY{n}{i}\PY{p}{:}\PY{p}{(}\PY{n}{data\PYZus{}pulsar}\PY{p}{[}\PY{n}{i}\PY{p}{]}\PY{p}{[}\PY{n}{idx\PYZus{}pk}\PY{p}{[}\PY{l+m+mi}{0}\PY{p}{]}\PY{p}{]}\PY{o}{\PYZhy{}}\PY{n}{data\PYZus{}pulsar}\PY{p}{[}\PY{n}{i}\PY{p}{]}\PY{p}{[}\PY{n}{idx\PYZus{}pk}\PY{p}{[}\PY{l+m+mi}{1}\PY{p}{]}\PY{p}{]}\PY{p}{)} \PY{k}{for} \PY{n}{i}\PY{p}{,}\PY{n}{idx\PYZus{}pk} \PY{o+ow}{in} \PY{n}{idx\PYZus{}pk\PYZus{}dict}\PY{o}{.}\PY{n}{items}\PY{p}{(}\PY{p}{)}\PY{p}{\PYZcb{}}
        
        \PY{c+c1}{\PYZsh{} The indices (in nu\PYZus{}vec) for which the max is sufficiently large relative to other times in the same frequency}
        \PY{n}{idx\PYZus{}pk\PYZus{}visible\PYZus{}vec} \PY{o}{=} \PY{p}{[}\PY{n}{i} \PY{k}{for} \PY{n}{i}\PY{p}{,}\PY{n}{v} \PY{o+ow}{in} \PY{n}{diff\PYZus{}pk\PYZus{}dict}\PY{o}{.}\PY{n}{items}\PY{p}{(}\PY{p}{)} \PY{k}{if} \PY{n}{v}\PY{o}{\PYZgt{}}\PY{o}{=}\PY{n}{diff\PYZus{}threshold\PYZus{}dict}\PY{p}{[}\PY{n}{i}\PY{p}{]}\PY{p}{]}
        
        \PY{n+nb}{print}\PY{p}{(}\PY{n+nb}{len}\PY{p}{(}\PY{n}{idx\PYZus{}pk\PYZus{}visible\PYZus{}vec}\PY{p}{)}\PY{p}{)}
        \PY{c+c1}{\PYZsh{} print(idx\PYZus{}pk\PYZus{}visible\PYZus{}vec)}
\end{Verbatim}


    \begin{Verbatim}[commandchars=\\\{\}]
7

    \end{Verbatim}

    Initial Estimate Method 1: Find the Time of Peak for a Given Frequency

    \begin{Verbatim}[commandchars=\\\{\}]
{\color{incolor}In [{\color{incolor}6}]:} \PY{n}{plt}\PY{o}{.}\PY{n}{plot}\PY{p}{(}\PY{n}{t\PYZus{}vec}\PY{p}{,} \PY{n}{data\PYZus{}pulsar}\PY{p}{[}\PY{o}{\PYZhy{}}\PY{l+m+mi}{1}\PY{p}{]}\PY{p}{)}
        \PY{n}{plt}\PY{o}{.}\PY{n}{xlabel}\PY{p}{(}\PY{l+s+s1}{\PYZsq{}}\PY{l+s+s1}{Time [s]}\PY{l+s+s1}{\PYZsq{}}\PY{p}{)}
        \PY{n}{plt}\PY{o}{.}\PY{n}{ylabel}\PY{p}{(}\PY{l+s+s1}{\PYZsq{}}\PY{l+s+s1}{Intensity [AU]}\PY{l+s+s1}{\PYZsq{}}\PY{p}{)}
        \PY{n}{plt}\PY{o}{.}\PY{n}{title}\PY{p}{(}\PY{n}{f}\PY{l+s+s1}{\PYZsq{}}\PY{l+s+s1}{Frequency = }\PY{l+s+s1}{\PYZob{}}\PY{l+s+s1}{nu\PYZus{}vec[\PYZhy{}1]*1e\PYZhy{}9\PYZcb{} GHz}\PY{l+s+s1}{\PYZsq{}}\PY{p}{)}
\end{Verbatim}


\begin{Verbatim}[commandchars=\\\{\}]
{\color{outcolor}Out[{\color{outcolor}6}]:} Text(0.5,1,'Frequency = 2.0 GHz')
\end{Verbatim}
            
    \begin{center}
    \adjustimage{max size={0.9\linewidth}{0.9\paperheight}}{output_7_1.png}
    \end{center}
    { \hspace*{\fill} \\}
    
    \begin{Verbatim}[commandchars=\\\{\}]
{\color{incolor}In [{\color{incolor}7}]:} \PY{c+c1}{\PYZsh{} Calculate a 2D array of dispersion measures DM where DM\PYZus{}ij = calculated dispersion}
        \PY{c+c1}{\PYZsh{} measure between the indices i \PYZam{} j of two valid times }
        \PY{c+c1}{\PYZsh{} (i.e. times which had peaks which met the criteria for being \PYZsq{}big enough\PYZsq{}).}
        \PY{c+c1}{\PYZsh{} Shave duplicates and the diagonal.}
        \PY{n}{num\PYZus{}valid} \PY{o}{=} \PY{n+nb}{len}\PY{p}{(}\PY{n}{idx\PYZus{}pk\PYZus{}visible\PYZus{}vec}\PY{p}{)}
        \PY{n}{DM\PYZus{}array} \PY{o}{=} \PY{n}{np}\PY{o}{.}\PY{n}{zeros}\PY{p}{(}\PY{p}{(}\PY{n}{num\PYZus{}valid}\PY{p}{,} \PY{n}{num\PYZus{}valid}\PY{p}{)}\PY{p}{)}\PY{o}{*}\PY{n}{np}\PY{o}{.}\PY{n}{NAN}
        \PY{k}{for} \PY{n}{i}\PY{p}{,}\PY{n}{idx0} \PY{o+ow}{in} \PY{n+nb}{enumerate}\PY{p}{(}\PY{n}{idx\PYZus{}pk\PYZus{}visible\PYZus{}vec}\PY{p}{)}\PY{p}{:}
            \PY{k}{for} \PY{n}{j}\PY{p}{,}\PY{n}{idx1} \PY{o+ow}{in} \PY{n+nb}{enumerate}\PY{p}{(}\PY{n}{idx\PYZus{}pk\PYZus{}visible\PYZus{}vec}\PY{p}{)}\PY{p}{:}
                \PY{k}{if} \PY{n}{i}\PY{o}{\PYZlt{}}\PY{o}{=}\PY{n}{j}\PY{p}{:}
                    \PY{k}{continue}
                \PY{n}{nu0} \PY{o}{=} \PY{n}{nu\PYZus{}vec}\PY{p}{[}\PY{n}{idx0}\PY{p}{]}
                \PY{n}{nu1} \PY{o}{=} \PY{n}{nu\PYZus{}vec}\PY{p}{[}\PY{n}{idx1}\PY{p}{]}
                
                \PY{c+c1}{\PYZsh{} Ignoring duplicated frequencies}
                \PY{k}{if} \PY{n}{nu0} \PY{o}{==} \PY{n}{nu1}\PY{p}{:}
                    \PY{k}{continue}
                    
                \PY{n}{t0} \PY{o}{=} \PY{n}{t\PYZus{}vec}\PY{p}{[}\PY{n}{np}\PY{o}{.}\PY{n}{argmax}\PY{p}{(}\PY{n}{data\PYZus{}pulsar}\PY{p}{[}\PY{n}{idx0}\PY{p}{]}\PY{p}{)}\PY{p}{]}
                \PY{n}{t1} \PY{o}{=} \PY{n}{t\PYZus{}vec}\PY{p}{[}\PY{n}{np}\PY{o}{.}\PY{n}{argmax}\PY{p}{(}\PY{n}{data\PYZus{}pulsar}\PY{p}{[}\PY{n}{idx1}\PY{p}{]}\PY{p}{)}\PY{p}{]}
                \PY{n}{DM\PYZus{}array}\PY{p}{[}\PY{n}{i}\PY{p}{]}\PY{p}{[}\PY{n}{j}\PY{p}{]} \PY{o}{=} \PY{n}{estimate\PYZus{}DM}\PY{p}{(}\PY{n}{nu0}\PY{p}{,} \PY{n}{nu1}\PY{p}{,} \PY{n}{t0}\PY{p}{,} \PY{n}{t1}\PY{p}{)}
        \PY{c+c1}{\PYZsh{}         print(f\PYZsq{}\PYZob{}\PYZdq{}\PYZob{}0:2e\PYZcb{}\PYZdq{}.format(nu0)\PYZcb{}/\PYZob{}\PYZdq{}\PYZob{}0:2e\PYZcb{}\PYZdq{}.format(nu1)\PYZcb{} \PYZhy{}\PYZgt{} \PYZob{}DM\PYZus{}array[i][j]\PYZcb{}\PYZsq{})}
        \PY{c+c1}{\PYZsh{}         if DM\PYZus{}array[i][j] \PYZgt{} 0:}
        \PY{c+c1}{\PYZsh{}             plt.figure()}
        \PY{c+c1}{\PYZsh{}             for n in (idx0, idx1):}
        \PY{c+c1}{\PYZsh{}                 plt.plot(t\PYZus{}vec, data\PYZus{}pulsar[n], label=f\PYZsq{}nu=\PYZob{}\PYZdq{}\PYZob{}0:.3f\PYZcb{}\PYZdq{}.format(nu\PYZus{}vec[n]*1e\PYZhy{}9)\PYZcb{} GHz\PYZsq{})}
        \PY{c+c1}{\PYZsh{}                 plt.legend()}
        \PY{c+c1}{\PYZsh{}                 plt.xlabel(\PYZsq{}Time [s]\PYZsq{})}
        \PY{c+c1}{\PYZsh{}                 plt.ylabel(\PYZsq{}Power [AU]\PYZsq{})}
        
        \PY{n}{pprint}\PY{p}{(}\PY{n}{DM\PYZus{}array}\PY{p}{)}
        \PY{n}{DM\PYZus{}array\PYZus{}frequency} \PY{o}{=} \PY{n}{np}\PY{o}{.}\PY{n}{array}\PY{p}{(}\PY{n}{DM\PYZus{}array}\PY{p}{,} \PY{n}{copy}\PY{o}{=}\PY{k+kc}{True}\PY{p}{)}
\end{Verbatim}


    \begin{Verbatim}[commandchars=\\\{\}]
array([[nan, nan, nan, nan, nan, nan, nan],
       [5.51e+20, nan, nan, nan, nan, nan, nan],
       [-1.57e+19, -5.00e+19, nan, nan, nan, nan, nan],
       [1.60e+21, 1.66e+21, 3.27e+22, nan, nan, nan, nan],
       [3.21e+20, 3.12e+20, 1.08e+21, -3.13e+21, nan, nan, nan],
       [6.14e+19, 4.50e+19, 1.63e+20, -2.23e+21, -1.12e+21, nan, nan],
       [6.59e+20, 6.63e+20, 1.55e+21, -7.40e+20, 2.19e+21, 4.64e+23, nan]])

    \end{Verbatim}

    Initial Estimate Method 2: Use Prior Knowledge (t=0 @ Start of Pulse
Arrival)

    \begin{Verbatim}[commandchars=\\\{\}]
{\color{incolor}In [{\color{incolor}8}]:} \PY{n}{t\PYZus{}pk\PYZus{}vec} \PY{o}{=} \PY{p}{[}\PY{n}{t\PYZus{}vec}\PY{p}{[}\PY{n}{np}\PY{o}{.}\PY{n}{argmax}\PY{p}{(}\PY{n}{data\PYZus{}pulsar}\PY{p}{[}\PY{n}{i}\PY{p}{]}\PY{p}{)}\PY{p}{]} \PY{k}{for} \PY{n}{i}\PY{p}{,}\PY{n}{\PYZus{}} \PY{o+ow}{in} \PY{n+nb}{enumerate}\PY{p}{(}\PY{n}{nu\PYZus{}vec}\PY{p}{)}\PY{p}{]}
        
        \PY{n}{plt}\PY{o}{.}\PY{n}{figure}\PY{p}{(}\PY{p}{)}
        \PY{n}{plt}\PY{o}{.}\PY{n}{plot}\PY{p}{(}\PY{n}{nu\PYZus{}vec}\PY{o}{*}\PY{l+m+mf}{1e\PYZhy{}9}\PY{p}{,} \PY{n}{t\PYZus{}pk\PYZus{}vec}\PY{p}{,} \PY{l+s+s1}{\PYZsq{}}\PY{l+s+s1}{o}\PY{l+s+s1}{\PYZsq{}}\PY{p}{)}
        \PY{n}{plt}\PY{o}{.}\PY{n}{xlabel}\PY{p}{(}\PY{l+s+s1}{\PYZsq{}}\PY{l+s+s1}{Frequency [GHz]}\PY{l+s+s1}{\PYZsq{}}\PY{p}{)}
        \PY{n}{plt}\PY{o}{.}\PY{n}{ylabel}\PY{p}{(}\PY{l+s+s1}{\PYZsq{}}\PY{l+s+s1}{Time of Peak [s]}\PY{l+s+s1}{\PYZsq{}}\PY{p}{)}
\end{Verbatim}


\begin{Verbatim}[commandchars=\\\{\}]
{\color{outcolor}Out[{\color{outcolor}8}]:} Text(0,0.5,'Time of Peak [s]')
\end{Verbatim}
            
    \begin{center}
    \adjustimage{max size={0.9\linewidth}{0.9\paperheight}}{output_10_1.png}
    \end{center}
    { \hspace*{\fill} \\}
    
    There's not enough rhyme or reason to the above to comfortably throw
something like least squares at the raw data.

    \begin{Verbatim}[commandchars=\\\{\}]
{\color{incolor}In [{\color{incolor}9}]:} \PY{n}{DM\PYZus{}vec\PYZus{}frequency} \PY{o}{=} \PY{p}{[}\PY{p}{]}
        \PY{k}{for} \PY{n}{idx} \PY{o+ow}{in} \PY{n}{idx\PYZus{}pk\PYZus{}visible\PYZus{}vec}\PY{p}{:}
            \PY{n}{nu0} \PY{o}{=} \PY{n}{np}\PY{o}{.}\PY{n}{max}\PY{p}{(}\PY{n}{nu\PYZus{}vec}\PY{p}{)}
            \PY{n}{nu1} \PY{o}{=} \PY{n}{nu\PYZus{}vec}\PY{p}{[}\PY{n}{idx}\PY{p}{]}
            \PY{n}{t0} \PY{o}{=} \PY{n}{t\PYZus{}vec}\PY{p}{[}\PY{n}{np}\PY{o}{.}\PY{n}{argmax}\PY{p}{(}\PY{n}{data\PYZus{}pulsar}\PY{p}{[}\PY{o}{\PYZhy{}}\PY{l+m+mi}{1}\PY{p}{]}\PY{p}{)}\PY{p}{]}
            \PY{n}{t1} \PY{o}{=} \PY{n}{t\PYZus{}vec}\PY{p}{[}\PY{n}{np}\PY{o}{.}\PY{n}{argmax}\PY{p}{(}\PY{n}{data\PYZus{}pulsar}\PY{p}{[}\PY{n}{idx}\PY{p}{]}\PY{p}{)}\PY{p}{]}
            \PY{n}{DM\PYZus{}vec\PYZus{}frequency}\PY{o}{.}\PY{n}{append}\PY{p}{(}\PY{n}{estimate\PYZus{}DM}\PY{p}{(}\PY{n}{nu0}\PY{p}{,} \PY{n}{nu1}\PY{p}{,} \PY{n}{t0}\PY{p}{,} \PY{n}{t1}\PY{p}{)}\PY{p}{)}
        \PY{n}{DM\PYZus{}vec\PYZus{}frequency} \PY{o}{=} \PY{n}{np}\PY{o}{.}\PY{n}{array}\PY{p}{(}\PY{n}{DM\PYZus{}vec\PYZus{}frequency}\PY{p}{)}
        \PY{n}{pprint}\PY{p}{(}\PY{n}{DM\PYZus{}vec\PYZus{}frequency}\PY{p}{)}
\end{Verbatim}


    \begin{Verbatim}[commandchars=\\\{\}]
array([7.53e+20, 7.60e+20, 1.68e+21, -3.97e+20, 2.37e+21, 1.82e+22,
       3.21e+21])

    \end{Verbatim}

    General curiosity: You could rotate this and treat it as tracking a
moving object (where the object's position is the the frequency of the
peak) with variable acceleration, then throw an extended Kalman filter
at it to figure out the DM with the initial estimate calculated using
either of the methods above

    2.2

    \begin{Verbatim}[commandchars=\\\{\}]
{\color{incolor}In [{\color{incolor}10}]:} \PY{n}{n\PYZus{}e} \PY{o}{=} \PY{o}{.}\PY{l+m+mi}{03} \PY{c+c1}{\PYZsh{} electrons/cm\PYZca{}2}
         
         \PY{c+c1}{\PYZsh{} Filtering out intentionally introduced NaNs and impossible DMs}
         \PY{c+c1}{\PYZsh{} caused by noise overwhelming the real pulse}
         \PY{n}{DM\PYZus{}vec} \PY{o}{=}  \PY{n}{DM\PYZus{}array\PYZus{}frequency}\PY{o}{.}\PY{n}{flatten}\PY{p}{(}\PY{p}{)} \PY{c+c1}{\PYZsh{} DM\PYZus{}array\PYZus{}frequency.flatten()/DM\PYZus{}vec\PYZus{}frequency}
         \PY{n}{DM\PYZus{}vec} \PY{o}{=} \PY{n}{DM\PYZus{}vec}\PY{p}{[}\PY{o}{\PYZti{}}\PY{n}{np}\PY{o}{.}\PY{n}{isnan}\PY{p}{(}\PY{n}{DM\PYZus{}vec}\PY{p}{)}\PY{p}{]}
         \PY{n}{DM\PYZus{}vec} \PY{o}{=} \PY{n}{DM\PYZus{}vec}\PY{p}{[}\PY{n}{DM\PYZus{}vec} \PY{o}{\PYZgt{}} \PY{l+m+mi}{0}\PY{p}{]}
         \PY{n}{DM} \PY{o}{=} \PY{n}{gmean}\PY{p}{(}\PY{n}{DM\PYZus{}vec}\PY{p}{)} \PY{c+c1}{\PYZsh{} np.mean/median(DM\PYZus{}vec)}
         \PY{n+nb}{print}\PY{p}{(}\PY{n}{f}\PY{l+s+s1}{\PYZsq{}}\PY{l+s+s1}{DM Estimate:}\PY{l+s+se}{\PYZbs{}t}\PY{l+s+se}{\PYZbs{}t}\PY{l+s+s1}{\PYZob{}}\PY{l+s+s1}{\PYZdq{}}\PY{l+s+si}{\PYZob{}0:2e\PYZcb{}}\PY{l+s+s1}{\PYZdq{}}\PY{l+s+s1}{.format(DM)\PYZcb{} electrons/cm\PYZca{}2}\PY{l+s+s1}{\PYZsq{}}\PY{p}{)}
         
         \PY{n}{d} \PY{o}{=} \PY{n}{DM}\PY{o}{/}\PY{n}{n\PYZus{}e} \PY{c+c1}{\PYZsh{} cm}
         \PY{n+nb}{print}\PY{p}{(}\PY{n}{f}\PY{l+s+s1}{\PYZsq{}}\PY{l+s+s1}{Distance Estimate:}\PY{l+s+se}{\PYZbs{}t}\PY{l+s+s1}{\PYZob{}}\PY{l+s+s1}{\PYZdq{}}\PY{l+s+si}{\PYZob{}0:2e\PYZcb{}}\PY{l+s+s1}{\PYZdq{}}\PY{l+s+s1}{.format(d)\PYZcb{} cm = }\PY{l+s+s1}{\PYZob{}}\PY{l+s+s1}{d*cm2pc\PYZcb{} pc}\PY{l+s+s1}{\PYZsq{}}\PY{p}{)}
\end{Verbatim}


    \begin{Verbatim}[commandchars=\\\{\}]
DM Estimate:		1.032229e+21 electrons/cm\^{}2
Distance Estimate:	3.440765e+22 cm = 11150.758494193946 pc

    \end{Verbatim}

    The value for distance seems high.

    2.3

    \begin{Verbatim}[commandchars=\\\{\}]
{\color{incolor}In [{\color{incolor}11}]:} \PY{n}{omega\PYZus{}p} \PY{o}{=} \PY{n}{np}\PY{o}{.}\PY{n}{sqrt}\PY{p}{(}\PY{l+m+mi}{4}\PY{o}{*}\PY{n}{np}\PY{o}{.}\PY{n}{pi} \PY{o}{*} \PY{n}{n\PYZus{}e} \PY{o}{*} \PY{n}{q}\PY{o}{*}\PY{o}{*}\PY{l+m+mi}{2} \PY{o}{/} \PY{n}{m\PYZus{}e}\PY{p}{)}
         \PY{n+nb}{print}\PY{p}{(}\PY{n}{f}\PY{l+s+s1}{\PYZsq{}}\PY{l+s+s1}{Plasma Frequency: }\PY{l+s+s1}{\PYZob{}}\PY{l+s+s1}{\PYZdq{}}\PY{l+s+si}{\PYZob{}0:2e\PYZcb{}}\PY{l+s+s1}{\PYZdq{}}\PY{l+s+s1}{.format(omega\PYZus{}p/(2*np.pi))\PYZcb{} Hz}\PY{l+s+s1}{\PYZsq{}}\PY{p}{)}
\end{Verbatim}


    \begin{Verbatim}[commandchars=\\\{\}]
Plasma Frequency: 1.545097e+03 Hz

    \end{Verbatim}

    2.4

    \begin{Verbatim}[commandchars=\\\{\}]
{\color{incolor}In [{\color{incolor}12}]:} \PY{n}{sigma\PYZus{}Thomson} \PY{o}{=} \PY{l+m+mi}{8}\PY{o}{*}\PY{n}{np}\PY{o}{.}\PY{n}{pi}\PY{o}{/}\PY{l+m+mi}{3} \PY{o}{*} \PY{p}{(}\PY{n}{q}\PY{o}{*}\PY{o}{*}\PY{l+m+mi}{2}\PY{o}{/}\PY{p}{(}\PY{n}{m\PYZus{}e}\PY{o}{*}\PY{n}{c}\PY{o}{*}\PY{o}{*}\PY{l+m+mi}{2}\PY{p}{)}\PY{p}{)}\PY{o}{*}\PY{o}{*}\PY{l+m+mi}{2}
         \PY{n}{alpha\PYZus{}Thomson} \PY{o}{=} \PY{n}{n\PYZus{}e} \PY{o}{*} \PY{n}{sigma\PYZus{}Thomson}
         \PY{n}{tau\PYZus{}Thomson} \PY{o}{=} \PY{n}{alpha\PYZus{}Thomson} \PY{o}{*} \PY{n}{d}
         \PY{n+nb}{print}\PY{p}{(}\PY{n}{f}\PY{l+s+s1}{\PYZsq{}}\PY{l+s+s1}{Optical Depth: }\PY{l+s+si}{\PYZob{}tau\PYZus{}Thomson\PYZcb{}}\PY{l+s+s1}{\PYZsq{}}\PY{p}{)}
\end{Verbatim}


    \begin{Verbatim}[commandchars=\\\{\}]
Optical Depth: 0.0006672518908541399

    \end{Verbatim}


    % Add a bibliography block to the postdoc
    
    
    
    \end{document}
