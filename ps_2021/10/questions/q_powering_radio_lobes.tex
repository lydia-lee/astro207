\qns{Powering Radio Lobes}

% \begin{itemize}
% 	% \item $L_\nu = 4\pi d^2F_\nu = 5(10^{36})\left(\frac{\nu}{10\si{\mega\hertz}}\right)^{-0.8} \frac{\text{\erg}}{\si{\second\cdot\hertz}}$ @ $d \approx 230$MPc
% 	\item $nd\gamma = C\gamma^p d\gamma, \gamma > \gamma_\text{min}$
% \end{itemize}

\begin{enumerate}

%%% 1
\qitem{}
\work{
	Using points $(40\si{\mega\hertz}, 2.3\cdot10^4\text{Jy})$ and $(2\cdot10^4\si{\mega\hertz}, 500\text{Jy})$ on the curve and using
	\begin{align*}
		D &\propto \nu^{\frac{1+p}{2}}\\
		\frac{\Delta(\log D)}{\Delta(\log\nu)} &= \frac{1+p}{2}
	\end{align*}
	(where $p < 0$)}
\ans{
	\centering
	\textcolor{red}{TODO numerical}}

%%% 2
\qitem{}
\work{
	\begin{align*}
		n\cdot d\gamma &= C\gamma^p\cdot d\gamma\\
		U_e &= \int_{\gamma_\text{min}}^\infty C\gamma^p\cdot d\gamma\\
			&= C\left(\frac{1}{1+p}\gamma^{1_p}\right)\rvert_{\gamma=\gamma_\text{min}}^{\gamma=\infty}\\
			&= -C\frac{\gamma_\text{min}^{1+p}}{1+p} \longleftarrow 1+p < 0
	\end{align*}

	% Pretty sure finding gamma min and C were meant as rhetorical questions (for this part)
	% Finding $\gamma_\text{min}$ requires knowledge of either the cyclotron frequency or the magnetic field.
	% \begin{align*}
	% 	\nu_m \approx 10\si{\mega\hertz} &\approx \gamma_\text{min}^2\nu_\text{cyc}\\
	% 	\nu_\text{cyc} &= \frac{e}{2\pi m_ec}B
	% \end{align*}

	% Similarly, finding $C$ requires knowledge of $U_B$ and $\nu_\text{cyc}$
	}
\ans{
	\centering
	\textcolor{red}{TODO}}

%%% 3
\qitem{}
\work{
	\begin{align*}
		\nu_\text{cyc} &= \frac{qB}{m_ec}\\
		U_B &= \frac{B^2}{8\pi}
	\end{align*}

	\begin{multicols}{2}
		\raggedcolumns
		\begin{align*}
			\nu_m &= \frac{3}{2}\gamma_\text{min}^2\nu_\text{cyc}\sin\alpha\\
			\gamma_\text{min} &= \sqrt{\frac{2\nu_m}{3\nu_\text{cyc}\sin\alpha}}\\
				&= A_1\nu_m^\frac{1}{2}B^{-\frac{1}{2}} \text{ where } A_1 \equiv \sqrt{\frac{2m_ec}{3q\sin\alpha}} 
		\end{align*}
		\vfill\columnbreak
		\begin{align*}
			L_\nu &\approx \frac{2}{3}C\frac{U_B\sigma_Tc}{\nu_\text{cyc}}\left(\frac{\nu}{\nu_\text{cyc}}\right)^\frac{1+p}{2}\times V\\
			C &\approx \frac{3L_m\nu_\text{cyc}^\frac{3+p}{2}}{2U_B\sigma_Tc\nu_m^\frac{1+p}{2}V}\\
				&= A_2\frac{L_mB^\frac{-1+p}{2}}{
				\nu_m^\frac{1+p}{2}V} \text{ where } A_2 \equiv \frac{12\pi\left(\frac{q}{m_ec}\right)^\frac{3+p}{2}}{\sigma_Tc}
		\end{align*}
	\end{multicols}
	\begin{align*}
		U_e &= \frac{-1}{1+p}C\gamma_\text{min}\\
			&= A\frac{L_mB^\frac{-2+p}{2}\nu_m^{-\frac{p}{2}}}{V}
	\end{align*}}
\ans{\textcolor{red}{check algebra}}

%%% 4
\qitem{}
\ans{
	\begin{align*}
		E &= 2V(U_e + U_B)\\
		\frac{dE}{dB} &= 2V\left(\frac{dU_e}{dB} + \frac{dU_B}{dB}\right)\\
			&= 2V\left(-\frac{3}{2}\frac{U_e}{B} + 2\frac{U_B}{B}\right)\\
			&= \frac{2V}{B}\left(-\frac{3}{2}U_e + 2U_B\right)
	\end{align*}
	Discounting nonphysical limits like $B = \infty$ and $B = 0$,
	\begin{align*}
		U_B &= \frac{3}{4}U_e
	\end{align*}}

%%% 5
\qitem{}
\work{}
\ans{}

%%% 6
\qitem{}
\work{}
\ans{}

%%% 7
\qitem{}
\work{}
\ans{}

\end{enumerate}