\qns{Powering Radio Lobes}

% \begin{itemize}
% 	% \item $L_\nu = 4\pi d^2F_\nu = 5(10^{36})\left(\frac{\nu}{10\si{\mega\hertz}}\right)^{-0.8} \frac{\text{\erg}}{\si{\second\cdot\hertz}}$ @ $d \approx 230$MPc
% 	\item $nd\gamma = C\gamma^p d\gamma, \gamma > \gamma_\text{min}$
% \end{itemize}

\begin{enumerate}

%%% 1
\qitem{}
\work{
	Using points $(10\si{\mega\hertz}, 10^5\text{Jy})$ and $(2\cdot10^4\si{\mega\hertz}, 500\text{Jy})$ on the curve and
	\begin{align*}
		F_\nu &\propto \nu^{\frac{1+p}{2}}\\
		\frac{\Delta(\log F_\nu)}{\Delta(\log\nu)} &= \frac{1+p}{2}
	\end{align*}
	(where $p < 0$)}
\ans{
	$$p \approx -2.4$$
	\centering
	which jives reasonably well with the fitted line (which gives $p \approx -2.6$)}

%%% 2
\qitem{}
\work{
	\begin{align*}
	    n\cdot d\gamma &= C\gamma^p\cdot d\gamma\\
	    U_e &= m_ec^2C\frac{1}{p+2}\gamma^{p+2}\rvert_{\gamma_\text{min}}^{\gamma_\text{max}} \longleftarrow E = \gamma m_ec^2\\
	        &= \frac{m_ec^2}{p+2}C\left(\gamma_\text{max}^{p+2} - \gamma_\text{min}^{p+2}\right)\\
	        &\approx -\frac{m_ec^2}{p+2}C\gamma_\text{min}^{p+2}
	\end{align*}}
\ans{
	\centering
	$$U_e = \frac{m_ec^2}{p+2}C\left(\gamma_\text{max}^{p+2} - \gamma_\text{min}^{p+2}\right) \approx -\frac{m_ec^2}{p+2}C\gamma_\text{min}^{p+2}$$}

%%% 3
\newpage
\qitem{}
\work{
	\begin{align*}
		\omega_\text{cyc} &= \frac{qB}{m_ec}\\
		U_B &= \frac{B^2}{8\pi}
	\end{align*}

	\begin{multicols}{2}
	    \raggedcolumns
	    Finding $\gamma_\text{min}$ in terms of $\omega_m$ and $B$
	    \begin{align*}
	        \omega_m &= \frac{3}{2}\gamma_\text{min}^2\omega_\text{cyc}\sin\alpha\\
	        \gamma_\text{min} &= \sqrt{\frac{\omega_m}{\omega_\text{cyc}}\frac{2}{3}\frac{1}{\sin\alpha}}\\
	            &= \sqrt{\frac{2\omega_mm_ec}{3\sin\alpha eB}}\\
	            &= \sqrt{\frac{2m_ec}{3\sin\alpha e}}\omega_m^\frac{1}{2}B^{-\frac{1}{2}}\\
	            &= A_1 \omega_m^\frac{1}{2}B^{-\frac{1}{2}} \longleftarrow A_1  \equiv \sqrt{\frac{2m_ec}{3\sin\alpha e}}
	    \end{align*}
	    Note that $e$ is the charge of an electron because no one can figure out what $q$ is across contexts.
	    \vfill\columnbreak
	    Finding $C$ in terms of $\omega_m$, $B$, $L_m$, and the volume $V$
	    \begin{align*}
	        L_\nu &\approx \frac{2}{3}C\frac{U_B\sigma_Tc}{\nu_\text{cyc}}\left(\frac{\nu}{\nu_\text{cyc}}\right)^\frac{1+p}{2}\times V\\
	            &= \frac{2(2\pi)^{-\frac{3+p}{2}}}{3}CU_B\sigma_Tc\omega^\frac{1+p}{2}\omega_\text{cyc}^{-\frac{3+p}{2}}\times V\\
	        C &\approx L_m\frac{3}{2(2\pi)^{-\frac{3+p}{2}}}\frac{\omega_\text{cyc}^\frac{3+p}{2}}{U_B\sigma_Tc\omega_m^\frac{1+p}{2}V}\\
	            &= L_m\frac{3}{2(2\pi)^{-\frac{3+p}{2}}}\frac{8\pi\left(\frac{eB}{m_ec}\right)^\frac{3+p}{2}}{B^2\sigma_Tc\omega_m^\frac{1+p}{2}V} \longleftarrow U_B = \frac{B^2}{8\pi}\\
	            &= \frac{12\pi}{(2\pi)^{-\frac{3+p}{2}}}L_m\frac{\left(\frac{e}{m_ec}\right)^\frac{3+p}{2}}{\sigma_TcV}\omega_m^{-\frac{1+p}{2}}B^\frac{p-1}{2}\\
	            &= A_2\frac{L_m}{V}\omega_m^{-\frac{1+p}{2}}B^{\frac{p-1}{2}} \longleftarrow A_2\equiv \frac{12\pi}{(2\pi)^{-\frac{3+p}{2}}}\frac{\left(\frac{e}{m_ec}\right)^\frac{3+p}{2}}{\sigma_Tc}
	    \end{align*}
	    and I'm not going to simplify the expression for $A_2$ because what's a computational cycle or two in this day and age?
	\end{multicols}
	Plugging things into the expression for $U_e$
	\begin{align*}
		U_e &= \frac{m_ec^2}{p+2}C\left(\gamma_\text{max}^{p+2} - \gamma_\text{min}^{p+2}\right)\\
			&\approx -\frac{m_ec^2}{p+2}C\gamma_\text{min}^{p+2}\\
			&= -\frac{m_ec^2}{p+2}\cdot A\frac{L_m}{V}\omega_m^{-\frac{1+p}{2}}B^{\frac{p-1}{2}} \cdot \left(A_1\omega_m^\frac{1}{2}B^{-\frac{1}{2}}\right)^{2+p}\\
			&= A\frac{L_m}{V}\nu_m^\frac{1}{2}B^{-\frac{3}{2}}
	\end{align*}
}

%%% 4
\qitem{\textcolor{white}{bloop}}
\ans{
	\begin{align*}
		E &= 2V(U_e + U_B)\\
		\frac{dE}{dB} &= 2V\left(\frac{dU_e}{dB} + \frac{dU_B}{dB}\right)\\
			&= 2V\left(-\frac{3}{2}\frac{U_e}{B} + 2\frac{U_B}{B}\right)\\
			&= \frac{2V}{B}\left(-\frac{3}{2}U_e + 2U_B\right)
	\end{align*}
	Discounting nonphysical limits like $B = \infty$ and $B = 0$,
	\begin{align*}
		U_B &= \frac{3}{4}U_e
	\end{align*}}

%%% 5
\qitem{}
\work{
	\begin{align*}
		U_B &= \frac{3}{4}U_e\\
		\frac{1}{8\pi}B^2 &= \frac{3}{4}A\frac{L_m\nu_m^\frac{1}{2}}{V}B^{-\frac{3}{2}}\\
		AL_m\frac{\nu_m^\frac{1}{2}}{V}6\pi &= B^\frac{7}{2}\\
		B &= \left(\frac{AL_m\nu_m^\frac{1}{2}}{V}\right)^\frac{2}{7}
	\end{align*}
	and then 
	\begin{align*}
		\gamma_\text{min} &\approx \sqrt{\frac{2\nu_m}{3\nu_\text{cyc}}}
	\end{align*}}
\ans{
	\centering
	$$B \approx 4\cdot 10^{-5}\text{G}$$
	$$\gamma_\text{min} \approx 100$$}

%%% 6
\qitem{}
\work{
	\begin{align*}
		E &= 2V (U_e + U_B)
	\end{align*}}
\ans{
	\centering
	$$E \approx 10^{61}\text{erg}$$
	}

%%% 7
\qitem{
	\begin{itemize}
		\item $E_\text{created} = \frac{1}{10}m_\text{food}c^2$
	\end{itemize}}
\ans{
	\centering
	$m_\text{food} \approx 10^{41} \si{gram}$

	If SMBHs are $10^5$ to $10^6$ times larger than the sun, that's roughly 100$\times$ more masssive than a SMBH, suggesting the SMBH hypothesis doesn't hold up.}

\end{enumerate}