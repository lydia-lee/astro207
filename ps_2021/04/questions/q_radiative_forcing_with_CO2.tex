\qns{Radiative Forcing with CO$\bm{_2}$}

\newcommand\cotwo{CO$_2$\,}

Carbon dioxide is one of the major gases driving the anthropogenic greenhouse effect (the dominant greenhouse gas is H$_2$O, but it is not anthropogenic. One of the most impactful transition in \cotwo for our climate is the 15\si{\micro\meter} line, which corresponds to the bending oscillation of the linear \cotwo molecule. The main line (there are rotational/vibrational branches that we'll discuss in a later class) has an Einstein A of about $A \approx 10\si{\second^{-1}}$

\begin{enumerate}

\qitem{
	Assume the gas temperature of the \cotwo is about the same as the Earth's blackbody (255\si{\kelvin}, and plot the line profile of this transition with both Doppler and intrinsic components. Use a logarithmic $y$-axis.)}
\work{}
\ans{}

\qitem{
	Model the Earth's upgoing emission as a 255\si{\kelvin} blackbody. Plot this spectrum, and overlay on it the spectrum including absorption from the 15\si{\micro\meter} line of CO$_2$. At line center, the optical depth of this transition in our atmosphere is about $\tau\approx 3$, so set the number density of \cotwo accordingly.
}
\work{}
\ans{}

\qitem{
	Calculate the flux of emission absorbed by \cotwo over the frequency interval where this transition is optically thick $(\tau > 1)$. Compare this to the amount of energy absorbed by \cotwo over the entire blackbody spectrum. What fraction of the total energy absorbed is in the optically thick regime?
}
\work{}
\ans{}

\qitem{
	Assume that, for all the flux absorbed by \cotwo, ahlf of it is radiated upward (i.e. it escapes the atmosphere) and half of it is radiated downward. The flux radiated downward is effectively subtracted from the total flux radiated off by the Earth's blackbody. To radiate off the same amount of flux as it did without \cotwo, by how many degrees must the blackbody of the Earth increase? This is called radiative forcing, and you have now roughly calculated the contribution of the 15\si{\micro\meter} line of \cotwo to the greenhouse effect. In real life, there is a whole ladder of such lines that all contribute to the greenhouse effect.
}
\work{}
\ans{}


\end{enumerate}